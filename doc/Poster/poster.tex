% An example document showing the use of the ucbposter, based on the
% infposter demo developed by M.Reddy (MXR), 24/2/97 and updated by
% Mary E Foster, 10/5/2002. 

\documentclass{ucbposter}

\let\AuthorColour\red

%http://www.inf.ed.ac.uk/systems/tex/infposter/infposter.html
%\epsfigure[<width-percent>]{<filename>}{<caption>}
% \begin{itemize}
%   \item An example bullet point
%   \begin{itemize}
%     \item Example
%   \end{itemize}
% \end{itemize}


\begin{document}

\title{Recognizing Strong Gravitational Lenses}
\author{Chris Davis and Andrew McLeod}
\email{cpd@stanford.edu}
%\homepage{http://radlab.cs.berkeley.edu/}
\makeposter

\section{Strong Gravitational Lenses}

\epsfigure[40]{Figures/lens.png}{Example
strong gravitational lens discovered by \textsc{Space Warps}.}

\begin{itemize}
\item{Strong lenses -- systems with possibly multiple magnified and
distorted images of background objects (sources) due to the deflection of light
by massive foreground objects (lenses) -- can be used as astrophysical tools to
probe mass distributions, magnify distant objects, and measure fundamental
cosmological parameters.
\item{The main problem with strong gravitational lenses is their rarity:
  \begin{itemize}
    \item{For
modern wide-field optical surveys which observe tens of millions of galaxies
over many thousand square degrees, one expects to find only several hundred
such systems.} \item{While strong lenses
are relatively easy to spot by eye, machine learning techniques have thus far
been unsuccessful in reliably identifying strong lenses and distinguishing them from
image artifacts (e.g. cosmic rays) and other false positives (e.g. pixel
noise). Maybe CNNs will help!}
\end{itemize}}
\end{itemize}
\epsfigure[90]{Figures/sims.png}{Example fields (with cutouts in corner)
containing simulated lenses.}

\epsfigure[90]{Figures/duds.png}{Example fields (with cutouts in corner)
containing no lenses. Cutouts indicate regions used as inputs for `duds'.}


\section{Space Warps Dataset}

\begin{itemize}
  \item{\textsc{Space Warps} (\url{www.spacewarps.com}) is an online crowd-sourced gravitational
    lens detection system that invites citizen
scientists to interpret real data from the Canada-France-Hawaii Telescope
Legacy Survey.}
\item{Volunteers also analyze artificial ``training'' images of simulated
    lenses as well as images confirmed to have no strong gravitational lenses.
    Simulated lenses are also further demarcated by lens type (lensed quasar,
  lensing galaxy, lensing cluster), each of which has different potential
cosmological applications.}
\item{\textsc{Space Warps} is divided into two stages, where the second stage
    refines the results of the first by using considerably more difficult
  training images. There are 24177 training images in Stage 1 and 1876 in Stage
2. We also have 9030 ``test'' images (cutouts whose status is not known but for
which we are interested in making reasonable guesses).}
\item{\textsc{Space Warps} will provide a ``curated'' dataset of gravitational
  lenses and lens-look-alikes to facilitate the training of future automated detection
algorithms.}
\item{We train our networks on the ``training'' images only, of which there are
  20743 ``duds'' and 5310 ``sims''. From fields of 440 x 440 x 3, we cut
  out 96 x 96 x 3 stamps based on the locations \textsc{Space Warps}
users clicked.}
\end{itemize}


\section{Our Own CNN}


\section{Transfer Learning with Galaxy Morphology}

%\epsfigure[40]{Figures/architecture.pdf}{Architecture of the convolution
%  neural network trained on the galaxy zoo morphologies. The feature vector we
%use comes from the fully connected layer.}
%
%\epsfigure[40]{Figures/roc_curve.pdf}{Receiver Operating Curves for the Random
%  Forest Transfer Learning and the \textsc{Space Warps} systems, divided by
%  stage.}

  \epsfigure[90]{Figures/arch_roc.png}{Architecture of the convolution
  neural network trained on the galaxy zoo morphologies and Receiver Operating Curves for the Random
  Forest Transfer Learning and the \textsc{Space Warps} systems, divided by
  stage. }

\begin{itemize}
  \item{Kaggle held a competition to classify galaxy morphology using galaxies
    from the Sloan Digital Sky Survey, a comparable but older survey to the
  Canada-France-Hawaii-Telescope. Many of the high-ranking competitors used
convolution neural networks.}
  \item{We examined how well we could transfer the learning of morphology to
    identifying strong gravitational lenses by taking a convolution neural
  network trained on the Kaggle Galaxy Zoo data by Ryan Keisler and running our cutouts through
to obtain a length 500 feature vector.}
  \item{We then trained these feature vectors on a random forest, which is an
    ensemble of decision trees wherein the order and number of feature vectors
  used in a given tree is random.}
\item{For comparison, we plot the results from citizen scientists on these two
  stages as well. While we do worse overall, our results were extremely quick
 to compute, while the \textsc{Space Warps} project involved the
efforts of tens of thousands over several months.}
\item{Nevertheless, our training results take citizen-scientist inputs. There
  is much potential in the future for machines and citizen-scientists to learn
from one another!}
\end{itemize}

\section{Future}

\begin{itemize}
  \item{Expand categorization procedure beyond binary lens-not to the specific
    types of lenses.}
  \item{Explore deeper types of transfer learning (e.g. retraining parts of the
    network).}
  \item{SOMETHING ELSE}
\end{itemize}

\section{Acknowledgements}

We received access to the dataset through Phil Marshall
(\url{pjm@slac.stanford.edu}), who also graciously explained to us how
\textsc{Space Warps}
currently works. The convolutional neural network upon which the transfer
learning is based was kindly provided to us by Ryan Keisler
(\url{rkeisler@stanford.edu}).
All errors in interpretation or otherwise are entirely our own.

\end{document}

