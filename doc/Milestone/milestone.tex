\documentclass[10pt,twocolumn,letterpaper]{article}

\usepackage{cvpr}
\usepackage{times}
\usepackage{epsfig}
\usepackage{graphicx}
\usepackage{amsmath}
\usepackage{amssymb}

% Include other packages here, before hyperref.

% If you comment hyperref and then uncomment it, you should delete
% egpaper.aux before re-running latex.  (Or just hit 'q' on the first latex
% run, let it finish, and you should be clear).
\usepackage[breaklinks=true,bookmarks=false]{hyperref}

\cvprfinalcopy % *** Uncomment this line for the final submission

\def\cvprPaperID{****} % *** Enter the CVPR Paper ID here
\def\httilde{\mbox{\tt\raisebox{-.5ex}{\symbol{126}}}}

% Pages are numbered in submission mode, and unnumbered in camera-ready
%\ifcvprfinal\pagestyle{empty}\fi
\setcounter{page}{4321}
\begin{document}

%%%%%%%%% TITLE
\title{Recognizing Strong Gravitational Lenses}

\author{Chris Davis\\
{\tt\small cpd@stanford.edu}
% For a paper whose authors are all at the same institution,
% omit the following lines up until the closing ``}''.
% Additional authors and addresses can be added with ``\and'',
% just like the second author.
% To save space, use either the email address or home page, not both
\and
Andrew McLeod\\
{\tt\small ajmcleod@stanford.edu}
}

\maketitle
%\thispagestyle{empty}

% %%%%%%%%% ABSTRACT
% \begin{abstract}
%    The ABSTRACT is to be in fully-justified italicized text, at the top
%    of the left-hand column, below the author and affiliation
%    information. Use the word ``Abstract'' as the title, in 12-point
%    Times, boldface type, centered relative to the column, initially
%    capitalized. The abstract is to be in 10-point, single-spaced type.
%    Leave two blank lines after the Abstract, then begin the main text.
%    Look at previous CVPR abstracts to get a feel for style and length.
% \end{abstract}

%%%%%%%%% BODY TEXT
\section{Introduction}

A consequence of Einstein's Theory of General Relativity is that mass bends the
path of light. When light passes through a particularly deep gravitational
potential, the deflections, called strong gravitational lenses, can produce
brilliant arcs and multiple images. The number and distribution of strong
lenses reflect the expansion history of the universe.

%-------------------------------------------------------------------------
\section{Problem Statement}

In this project we will use images collected by the Canada-France-Hawaii
Telescope Legacy Survey to analyze how Convolution Neural Networks can improve
automated detection of strong lens systems. From other graduate work (but not
coursework), we have the locations and categories of around one hundred and
twenty known strong lenses, three thousand large fields verified to contain no
strong lenses, six thousand simulated strong lenses, and several thousand
classifications by citizen-scientists of other potential strong lens systems.
These will form the core of our training and testing datasets; our metric will
be how well a CNN correctly identifies known and simulated lenses and non-lens
systems.

We would like to examine the following questions:
\begin{itemize}
\item{ Do we have enough data to reasonably train and test a CNN? Can we get
       around this by artificially inflating the data, e.g. by adding rotated
     images?}
\item{ What processing needs to be done on the data? What kind of scaling of
       pixel data is appropriate for automated detection? Should we compress
     the five different 'colors' to a smaller number of dimensions?}
   \item{ How do citizen-scientists do compared with this automated system?}
   \item{Can we use the results of citizen-scientists to train the CNN?}
\item{What sets of classifications are needed? Are we better served sticking
       to 'lens' and 'not lens', or should we use several classification
       categories ('lensed arcs', 'lensed multiple images', 'non-lens pixel
     noise')?}
\end{itemize}

%-------------------------------------------------------------------------
\section{Technical Approach}

From approximately 12000 fields of $440\times440\times3$ fields, we have
constructed ______ cutouts sized $96\times96\times3$. We 

%-------------------------------------------------------------------------
\section{Intermediate/Preliminary Results}


% {\small
% \bibliographystyle{ieee}
% \bibliography{egbib}
% }

\end{document}
